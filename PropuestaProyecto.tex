% !TEX TS-program = pdflatex
% !TEX encoding = UTF-8 Unicode

% This is a simple template for a LaTeX document using the "article" class.
% See "book", "report", "letter" for other types of document.

%\documentclass[11pt]{article} % use larger type; default would be 10pt
\documentclass[a4paper]{article}

\usepackage[utf8]{inputenc} % set input encoding (not needed with XeLaTeX)
\usepackage[spanish]{babel}

%%% Examples of Article customizations
% These packages are optional, depending whether you want the features they provide.
% See the LaTeX Companion or other references for full information.

%%% PAGE DIMENSIONS
\usepackage{geometry} % to change the page dimensions
\geometry{a4paper} % or letterpaper (US) or a5paper or....
% \geometry{margin=2in} % for example, change the margins to 2 inches all round
% \geometry{landscape} % set up the page for landscape
%   read geometry.pdf for detailed page layout information

\usepackage{graphicx} % support the \includegraphics command and options

% \usepackage[parfill]{parskip} % Activate to begin paragraphs with an empty line rather than an indent

%%% PACKAGES
%\usepackage{booktabs} % for much better looking tables
%\usepackage{array} % for better arrays (eg matrices) in maths
%\usepackage{paralist} % very flexible & customisable lists (eg. enumerate/itemize, etc.)
%\usepackage{verbatim} % adds environment for commenting out blocks of text & for better verbatim
\usepackage{subfig} % make it possible to include more than one captioned figure/table in a single float
% These packages are all incorporated in the memoir class to one degree or another...

%%% HEADERS & FOOTERS
%\usepackage{fancyhdr} % This should be set AFTER setting up the page geometry
%\pagestyle{fancy} % options: empty , plain , fancy
%\renewcommand{\headrulewidth}{0pt} % customise the layout...
%\lhead{}\chead{}\rhead{}
%\lfoot{}\cfoot{\thepage}\rfoot{}

%%% SECTION TITLE APPEARANCE
\usepackage{sectsty}
\allsectionsfont{\sffamily\mdseries\upshape} % (See the fntguide.pdf for font help)
\usepackage{titling}
\newcommand{\subtitle}[1]{%
  \posttitle{%
    \par\end{center}
    \begin{center}\large#1\end{center}
    \vskip0.5em}%
}
% (This matches ConTeXt defaults)

%%% ToC (table of contents) APPEARANCE
\usepackage[nottoc,notlof,notlot]{tocbibind} % Put the bibliography in the ToC
\usepackage[titles,subfigure]{tocloft} % Alter the style of the Table of Contents
\renewcommand{\cftsecfont}{\rmfamily\mdseries\upshape}
\renewcommand{\cftsecpagefont}{\rmfamily\mdseries\upshape} % No bold!

%%% END Article customizations

%%% The "real" document content comes below...

\begin{document}

\title{Propuesta de Proyecto\\Problemas Especiales de Ingeniería en Computación}
\subtitle{Simulación de Piloteo de Drones con Realidad Aumentada Utilizando Sensores para Configuración del Ambiente Virtual}

\author{Aníbla Vásquez Clad\\
\texttt{aniavasq@espol.edu.ec}
\and Peter Arcentales}

\maketitle

\section{Introducción y Justificación}
El uso de Drones o Vehículos Aéreos No Tripulados inicialmente con aplicaciones meramente militares se ha extendido a diversas áreas de la ciencia, este uso es muy frecuente en lugares donde una aeronave común tripulada difícilmente puede llegar, ya sea por factores ambientales o de espacio, o en circunstancias de alto riesgo. Los Drones pueden ser aplicados para tomar muestras, reconocimiento cartográfico e hidrológico, control y vigilancia y otras posibilidades no necesariamente militares.
El objetivo de este proyecto es crear una aplicación en la cual un usuario de un samrtphone o tablet pueda experimentar de forma virtual el uso de un Drone con la ayuda de realidad aumentada y sensores del dispositivo como GPS y acelerómetro para generar un ambiente sobre el cual el drone virtualmente se desplaza; aunque hoy en día el costo de un Drone con mera aplicación de vigilancia es comerciado, sus precios son considerablemente altos y un usuario no puede realizar una prueba de este sin antes comprarlo.
\section{Marco Teórico}
\section{Análisis}
\section{Recursos}
\section{Apéndice}

%content

%\subsection{A subsection}

%More text before content table.

\end{document}
